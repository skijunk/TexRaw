%%%%%%%%%%%%%%%%%%%%%%%%%%%%%%%%%%%%%%
% gBriefRAW.tex von skijunk@gmx.de ist lizenziert 					%
% unter einer Creative Commons Namensnennung - 				%
% Weitergabe unter gleichen Bedingungen 4.0 International Lizenz.		%
%%%%%%%%%%%%%%%%%%%%%%%%%%%%%%%%%%%%%%

\documentclass[12pt]{g-brief}
\usepackage[utf8]{inputenc}

\lochermarke
\faltmarken
\fenstermarken
\trennlinien
\klassisch

\Name                {MeineName}
\Strasse             {MeineStrasse}
\Zusatz              {MeinAdressZusatz}
\RetourAdresse       {MeinName, MeineStra\UTF{00DF}e, MeinePlz Stadt}
\Ort                 { MeinePlz Stadt}
\Land                {MeinLand}

\Telefon             {+49\ 000\ 00000}
\Telefax             {+49\ 000\ 00000}
\HTTP                {MeineWebsite}
\EMail               {Mine@myHost.com}

%% BLZ->BIC
%% Kontonr->Iban

\renewcommand{\blztext}{\footnotesize BIC}
\renewcommand{\kontotext}{\footnotesize IBAN}

\Bank                {MeineBank}
\Konto               {000.000.000}
\BLZ                 {000.000.00}

\Unterschrift        {MeineUnterschrift}

\Postvermerk         {z. B. E I N S C H R E I B E N}
\Adresse             {zuH\UTF{00E4}nden\\
                      Firma\\
                      Stra\UTF{00DF}e Nr.\\
                      Postfach\\
                      Land \\
                      Plz Stadt}

\Betreff             {Betreff}

\Datum               {\today}
\IhrZeichen          {IhrZeichen}
\IhrSchreiben        {IhrSchreiben}
\MeinZeichen         {MeinZeichen}

\Anrede              {Sehr geehrte Damen und Herren\\
                      Sehr geehrte Frau\\
                      Sehr geehrter Herr}
\Gruss               {Mit freundlichen Gr\UTF{00FC}\UTF{00DF}en}{1cm} %%Abstand kann umgeschrieben werden

\Anlagen             {Anlage}
\Verteiler           {Verteiler}

\begin{document}
\begin{g-brief}
Jemand musste Josef K. verleumdet haben, denn ohne dass er etwas B\UTF{00F6}ses getan
h\UTF{00E4}tte, wurde er eines Morgens verhaftet. \UTF{00BB}Wie ein Hund!\UTF{00AB} sagte er, es war,
als sollte die Scham ihn \UTF{00FC}berleben. Als Gregor Samsa eines Morgens aus
unruhigen Tr\UTF{00E4}umen erwachte, fand er sich in seinem Bett zu einem ungeheueren
Ungeziefer verwandelt. Und es war ihnen wie eine Best\UTF{00E4}tigung ihrer neuen Tr\UTF{00E4}ume
und guten Absichten, als am Ziele ihrer Fahrt die Tochter als erste sich erhob und
ihren jungen K\UTF{00F6}rper dehnte. \UTF{00BB}Es ist ein eigent\UTF{00FC}mlicher Apparat\UTF{00AB}, sagte der Offizier
zu dem Forschungsreisenden und \UTF{00FC}berblickte mit einem gewisserma\UTF{00DF}en bewundernden
Blick den ihm doch wohlbekannten Apparat. Sie h\UTF{00E4}tten noch ins Boot springen
k\UTF{00F6}nnen, aber der Reisende hob ein schweres, geknotetes Tau vom Boden, drohte ihnen
damit und hielt sie dadurch von dem Sprunge ab. In den letzten Jahrzehnten
ist das Interesse an Hungerk\UTF{00FC}nstlern sehr zur\UTF{00FC}ckgegangen.
Aber sie \UTF{00FC}berwanden sich, umdr\UTF{00E4}ngten den K\UTF{00E4}fig und wollten sich gar nicht fortr\UTF{00FC}hren.
Jemand musste Josef K. verleumdet haben, denn ohne dass er etwas B\UTF{00F6}ses getan h\UTF{00E4}tte,
wurde er eines Morgens verhaftet. \UTF{00BB}Wie ein Hund!\UTF{00AB} sagte er, es war,
als sollte die Scham ihn \UTF{00FC}berleben.
Als Gregor Samsa eines Morgens aus unruhigen Tr\UTF{00E4}umen erwachte, fand er sich
\end{g-brief}
\end{document}


\endinput
