\documentclass[11pt,ngerman,utf8]{g-brief}

\lochermarke
\faltmarken
\fenstermarken
\trennlinien
\klassisch

\Name                {MeineName}
\Strasse             {MeineStrasse}
\Zusatz              {}
\RetourAdresse       {}
\Ort                 {99533 Weitewelt}
\Land                {Deutschland}

\Telefon             {+49\ 000\ 00000}
\Telefax             {+49\ 000\ 00000}
\Telex               {}
\HTTP                {http://www.example.com}
\EMail               {raffzahn@example.com}

\Bank                {Bankhaus Skrupellos \& Co.}
\BLZ                 {000.000.00}
\Konto               {000.000.000}

\Unterschrift        {Otto Raffzahn}

\Postvermerk         {E I N S C H R E I B E N}
\Adresse             {Frau\\
                      Else Mittellos\\
                      Finsterweg 12\\
                      \\
                      D-93456 Tiefeschlucht
                      }

\Betreff             {Zahlung Ihrer Miete}

\Datum               {\today}
\IhrZeichen          {}
\IhrSchreiben        {}
\MeinZeichen         {OR}

\Anrede              {Sehr geehrte Frau Mittellos,}
\Gruss               {Mit freundlichen Gr\"u\ss{}en}{1cm}

\Anlagen             {}
\Verteiler           {}

\begin{document}
\begin{g-brief}
Jemand musste Josef K. verleumdet haben, denn ohne dass er etwas Böses getan
hätte, wurde er eines Morgens verhaftet. »Wie ein Hund!« sagte er, es war,
als sollte die Scham ihn überleben. Als Gregor Samsa eines Morgens aus
unruhigen Träumen erwachte, fand er sich in seinem Bett zu einem ungeheueren
Ungeziefer verwandelt. Und es war ihnen wie eine Bestätigung ihrer neuen Träume
und guten Absichten, als am Ziele ihrer Fahrt die Tochter als erste sich erhob und
ihren jungen Körper dehnte. »Es ist ein eigentümlicher Apparat«, sagte der Offizier
zu dem Forschungsreisenden und überblickte mit einem gewissermaßen bewundernden
Blick den ihm doch wohlbekannten Apparat. Sie hätten noch ins Boot springen
können, aber der Reisende hob ein schweres, geknotetes Tau vom Boden, drohte ihnen
damit und hielt sie dadurch von dem Sprunge ab. In den letzten Jahrzehnten
ist das Interesse an Hungerkünstlern sehr zurückgegangen.
Aber sie überwanden sich, umdrängten den Käfig und wollten sich gar nicht fortrühren.
Jemand musste Josef K. verleumdet haben, denn ohne dass er etwas Böses getan hätte,
wurde er eines Morgens verhaftet. »Wie ein Hund!« sagte er, es war,
als sollte die Scham ihn überleben.
Als Gregor Samsa eines Morgens aus unruhigen Träumen erwachte, fand er sich
\end{g-brief}
\end{document}


\endinput
